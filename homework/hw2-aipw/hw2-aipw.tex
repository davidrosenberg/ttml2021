%% LyX 2.3.6 created this file.  For more info, see http://www.lyx.org/.
%% Do not edit unless you really know what you are doing.
\documentclass[ruled]{article}
\usepackage{courier}
\usepackage[T1]{fontenc}
\usepackage[latin9]{inputenc}
\usepackage[letterpaper]{geometry}
\geometry{verbose}
\setcounter{secnumdepth}{5}
\usepackage{color}
\usepackage{enumitem}
\usepackage{algorithm2e}
\usepackage{dsfont}
\usepackage{amsmath}
\usepackage{amssymb}
\usepackage[unicode=true,
 bookmarks=false,
 breaklinks=false,pdfborder={0 0 1},backref=section,colorlinks=true]
 {hyperref}

\makeatletter

%%%%%%%%%%%%%%%%%%%%%%%%%%%%%% LyX specific LaTeX commands.
\providecommand{\LyX}{\texorpdfstring%
  {L\kern-.1667em\lower.25em\hbox{Y}\kern-.125emX\@}
  {LyX}}

%%%%%%%%%%%%%%%%%%%%%%%%%%%%%% Textclass specific LaTeX commands.
\newlength{\lyxlabelwidth}      % auxiliary length 

\@ifundefined{date}{}{\date{}}
%%%%%%%%%%%%%%%%%%%%%%%%%%%%%% User specified LaTeX commands.
\definecolor{mygreen}{rgb}{0,0.6,0}
\definecolor{mygray}{rgb}{0.5,0.5,0.5}
\definecolor{mymauve}{rgb}{0.58,0,0.82}

\makeatother

\usepackage{listings}
\lstset{backgroundcolor={\color{white}},
basicstyle={\footnotesize\ttfamily},
breakatwhitespace=false,
breaklines=true,
captionpos=b,
commentstyle={\color{mygreen}},
deletekeywords={...},
escapeinside={\%*}{*)},
extendedchars=true,
frame=shadowbox,
keepspaces=true,
keywordstyle={\color{blue}},
language=Python,
morekeywords={*,...},
numbers=none,
numbersep=5pt,
numberstyle={\tiny\color{mygray}},
rulecolor={\color{black}},
showspaces=false,
showstringspaces=false,
showtabs=false,
stepnumber=1,
stringstyle={\color{mymauve}},
tabsize=2}
\begin{document}
\global\long\def\reals{\mathbf{\mathbb{R}}}%
\global\long\def\integers{\mathbf{Z}}%
\global\long\def\naturals{\mathbf{N}}%
\global\long\def\rationals{\mathbf{Q}}%
\global\long\def\ca{\mathcal{A}}%
\global\long\def\cb{\mathcal{B}}%
\global\long\def\cc{\mathcal{C}}%
\global\long\def\cd{\mathcal{D}}%
\global\long\def\ce{\mathcal{E}}%
\global\long\def\cf{\mathcal{F}}%
\global\long\def\cg{\mathcal{G}}%
\global\long\def\ch{\mathcal{H}}%
\global\long\def\ci{\mathcal{I}}%
\global\long\def\cj{\mathcal{J}}%
\global\long\def\ck{\mathcal{K}}%
\global\long\def\cl{\mathcal{L}}%
\global\long\def\cm{\mathcal{M}}%
\global\long\def\cn{\mathcal{N}}%
\global\long\def\co{\mathcal{O}}%
\global\long\def\cp{\mathcal{P}}%
\global\long\def\cq{\mathcal{Q}}%
\global\long\def\calr{\mathcal{R}}%
\global\long\def\cs{\mathcal{S}}%
\global\long\def\ct{\mathcal{T}}%
\global\long\def\cu{\mathcal{U}}%
\global\long\def\cv{\mathcal{V}}%
\global\long\def\cw{\mathcal{W}}%
\global\long\def\cx{\mathcal{X}}%
\global\long\def\cy{\mathcal{Y}}%
\global\long\def\cz{\mathcal{Z}}%
\global\long\def\pr{\mathbb{P}}%
\global\long\def\predsp{\cy}%
\global\long\def\outsp{\cy}%
\global\long\def\prxy{P_{\cx\times\cy}}%
\global\long\def\prx{P_{\cx}}%
\global\long\def\prygivenx{P_{\cy\mid\cx}}%
\global\long\def\ex{\mathbb{E}}%
\global\long\def\var{\textrm{Var}}%
\global\long\def\cov{\textrm{Cov}}%
\global\long\def\sgn{\textrm{sgn}}%
\global\long\def\sign{\textrm{sign}}%
\global\long\def\kl{\textrm{KL}}%
\global\long\def\law{\mathcal{L}}%
\global\long\def\eps{\varepsilon}%
\global\long\def\as{\textrm{ a.s.}}%
\global\long\def\io{\textrm{ i.o.}}%
\global\long\def\ev{\textrm{ ev.}}%
\global\long\def\convd{\stackrel{d}{\to}}%
\global\long\def\convp{\stackrel{P}{\to}}%
\global\long\def\eqd{\stackrel{d}{=}}%
\global\long\def\del{\nabla}%
\global\long\def\loss{V}%
\global\long\def\risk{R}%
\global\long\def\etal{\textrm{et. al.}}%
\global\long\def\tr{\operatorname{tr}}%
\global\long\def\trace{\operatorname{trace}}%
\global\long\def\diag{\text{diag}}%
\global\long\def\rank{\text{rank}}%
\global\long\def\linspan{\text{span}}%
\global\long\def\proj{\text{Proj}}%
\global\long\def\argmax{\operatornamewithlimits{arg\, max}}%
\global\long\def\argmin{\operatornamewithlimits{arg\, min}}%
\global\long\def\bfx{\mathbf{x}}%
\global\long\def\bfy{\mathbf{y}}%
\global\long\def\bfl{\mathbf{\lambda}}%
\global\long\def\bfm{\mathbf{\mu}}%
\global\long\def\calL{\mathcal{L}}%
\global\long\def\vw{\boldsymbol{w}}%
\global\long\def\vx{\boldsymbol{x}}%
\global\long\def\vxi{\boldsymbol{\xi}}%
\global\long\def\valpha{\boldsymbol{\alpha}}%
\global\long\def\vbeta{\boldsymbol{\beta}}%
\global\long\def\vsigma{\boldsymbol{\sigma}}%
\global\long\def\vmu{\boldsymbol{\mu}}%
\global\long\def\vtheta{\boldsymbol{\theta}}%
\global\long\def\vd{\boldsymbol{d}}%
\global\long\def\vs{\boldsymbol{s}}%
\global\long\def\vt{\boldsymbol{t}}%
\global\long\def\vh{\boldsymbol{h}}%
\global\long\def\ve{\boldsymbol{e}}%
\global\long\def\vf{\boldsymbol{f}}%
\global\long\def\vg{\boldsymbol{g}}%
\global\long\def\vr{\boldsymbol{r}}%
\global\long\def\vz{\boldsymbol{z}}%
\global\long\def\vk{\boldsymbol{k}}%
\global\long\def\va{\boldsymbol{a}}%
\global\long\def\vb{\boldsymbol{b}}%
\global\long\def\vv{\boldsymbol{v}}%
\global\long\def\vy{\boldsymbol{y}}%
\global\long\def\hil{\ch}%
\global\long\def\rkhs{\hil}%
\global\long\def\median{\text{Median}}%
\global\long\def\ipw{\text{ipw}}%
\global\long\def\ipwsn{\ensuremath{\text{ipwsn}}}%
\global\long\def\iw{\text{iw}}%
\global\long\def\myfoot#1{\let\thefootnote\relax\footnotetext{\tiny{#1}}}%
% Causal inference
\global\long\def\ate{\text{ATE}}%
\global\long\def\unif{\text{Unif}}%
\global\long\def\bern{\text{Bern}}%
\global\long\def\indep{\perp\!\!\! \perp}%
\global\long\def\ind#1{\mathds{1}\left[#1\right]}%
\global\long\def\se#1{\text{SE}\left(#1\right)}%
\global\long\def\corr{\text{Corr}}%
\global\long\def\expit{\text{expit}}%
\global\long\def\snipw{\ensuremath{\text{sn\_ipw}}}%
\global\long\def\bias{\ensuremath{\text{Bias}}}%
\global\long\def\sd{\textrm{SD}}%
\global\long\def\opt{\text{opt}}%
\global\long\def\dm{\ensuremath{\text{dm}}}%
\global\long\def\poem{\text{POEM}}%


\title{Tools and Techniques for Machine Learning\\
Homework 2: Regression imputation, covariate shift, and control variates}

\maketitle
\textbf{Instructions}: Your answers to the questions below, including
plots and mathematical work, should be submitted as a single PDF file.
It's preferred that you write your answers using software that typesets
mathematics (e.g. \LaTeX , \LyX , or Jupyter), though if you need
to you may scan handwritten work. For submission, you can also export
your Jupyter notebook and merge that PDF with your PDF for the written
solutions into one file. \textbf{Don't forget to complete the Jupyter
notebook as well, for the programming part of this assignment}.

\section*{General hint on Adam's Law}

A couple times in this assignment we'll need a variant on the basic
Adam's Law. Adam's Law is that $\ex\left[\ex\left[Y\mid X\right]\right]=\ex Y$.
The variant we'll need is that Adam's Law still holds when everything
is conditioned on a particular event. For example, $\ex\left[\ex\left[Y\mid X,Z>a\right]\mid Z>a\right]=\ex\left[Y\mid Z>a\right]$.
We could see this by defining $\left(X',Y'\right)$ to have joint
distribution that's equal to the conditional distribution of $(X,Y)\mid Z>a$.
Then 
\begin{eqnarray*}
 &  & \ex\left[\ex\left[Y\mid X,Z>a\right]\mid Z>a\right]\\
 & = & \ex\left[\ex\left[Y'\mid X'\right]\right]=\ex\left[Y'\right]=\ex\left[Y\mid Z>a\right].
\end{eqnarray*}
 Another approach would be to define the random variable $W=\ind{Z>a}$.
Then
\[
\ex\left[\ex\left[Y\mid X,W\right]\mid W\right]=\ex\left[Y\mid W\right],
\]
by the generalized form of Adam's Law. This implies that
\[
\ex\left[\ex\left[Y\mid X,W=1\right]\mid W=1\right]=\ex\left[Y\mid W=1\right],
\]


\section{Complete case mean is unbiased for MCAR (when it's defined)}

Let $R_{i}\in\left\{ 0,1\right\} $ be the response indicator, $Y_{i}\in\reals$
the response. Consider the MCAR setting, in which $R_{i}\indep Y_{i}$,
and suppose $(R,Y),(R_{1},Y_{1}),\ldots,(R_{n},Y_{n})$ are i.i.d.
We observe data $\cd=\left(\left(R_{1},R_{1}Y_{1}\right),\ldots,\left(R_{n},R_{n}Y_{n}\right)\right)$.
The complete case estimator is defined as
\[
\hat{\mu}_{\text{cc}}=\hat{\mu}_{\text{cc}}(\cd)=\frac{\sum_{i=1}^{n}R_{i}Y_{i}}{\sum_{i=1}^{n}R_{i}}.
\]
Note that if $R_{1}=\cdots=R_{n}=0$, then $\hat{\mu}_{\text{cc}}=\frac{0}{0}$,
which is undefined. Since $\hat{\mu}_{\text{cc}}$ is undefined with
nonzero probability, it doesn't have an expectation or bias. In this
problem, we consider whether $\hat{\mu}_{\text{cc}}$ is unbiased
after ruling out the case where it's undefined.
\begin{enumerate}
\item Show that $\ex\left[\hat{\mu}_{\text{cc}}\mid\sum_{i=1}^{n}R_{i}>0\right]=\ex Y$.
(Hint: Show that $\ex\left[\hat{\mu}_{\text{cc}}\mid R_{1},\ldots,R_{n},\sum_{i=1}^{n}R_{i}>0\right]=\ex Y$.)

\end{enumerate}

\section{Regression imputation with $\protect\ex\left[Y\mid X=x\right]$ }

Consider the MAR setting. Let $\hat{f}(x)$ be a regression function
fit to the complete cases. Then the regression imputation estimator
for $\ex Y$ that we defined in class is given by
\[
\hat{\mu}_{\hat{f}}:=\frac{1}{n}\sum_{i=1}^{n}\left[R_{i}Y_{i}+\left(1-R_{i}\right)\hat{f}(X_{i})\right].
\]
There is an alternative form of regression imputation where we apply
$\hat{f}(x)$ to all the $X_{i}$'s, not just the incomplete cases.
This estimator is given by
\[
\hat{\mu}_{\hat{f}\text{-full}}:=\frac{1}{n}\sum_{i=1}^{n}\hat{f}(X_{i}).
\]
In this problem, we will verify that if we use $\ex\left[Y\mid X=x\right]$
for our regression imputation, then both of these regression imputation
estimators are unbiased. This will give us some hope that, under the
appropriate technical conditions, if our model is well-specified,
then each method of regression imputation is consistent.
\begin{enumerate}
\item If $f(x)=\ex\left[Y\mid X=x\right],$show that $\ex\left[\hat{\mu}_{f\text{-full}}\right]=\ex Y$.

\item If $f(x)=\ex\left[Y\mid X=x\right],$show that $\ex\left[\hat{\mu}_{f}\right]=\ex Y$.
(Hint: See the slide on ``Adam's Law / Law of iterated expectation''
for inspiration, and you'll also need to use the MAR assumption that
$Y_{i}\indep R_{i}\mid X_{i}$.)

 
\item If we expand out the two forms of the imputation estimator for a particular
set of observed data, we might get something like
\begin{eqnarray*}
\hat{\mu}_{\hat{f}} & = & \frac{1}{n}\left(Y_{1}+Y_{2}+\hat{f}(X_{3})+Y_{4}+\hat{f}(X_{5})+\cdots+Y_{n}\right)\\
\hat{\mu}_{\hat{f}\text{-full}} & = & \frac{1}{n}\left(\hat{f}(X_{1})+\hat{f}(X_{2})+\hat{f}(X_{3})+\hat{f}(X_{4})+\hat{f}(X_{5})+\cdots+\hat{f}(X_{n})\right).
\end{eqnarray*}
Written in this way, it's easy to see that $\hat{\mu}_{\hat{f}\text{-full}}$
and $\hat{\mu}_{\hat{f}}$ differ only in how they handle the complete
cases. We generally do not expect to have $\hat{f}(X_{i})=Y_{i}$
for all the compete cases -- that would indicate overfitting of our
imputation function. Nevertheless, you might be surprised to learn
that $\hat{\mu}_{\hat{f}}=\hat{\mu}_{\hat{f}\text{-full}}$ in some
common scenarios. Show that $\hat{\mu}_{\hat{f}\text{-full}}=\hat{\mu}_{\hat{f}}$
if we fit the regression function $\hat{f}(x)$ to the complete cases
using a linear model with intercept: 
\[
\hat{f}=\argmin_{\left\{ f:f(x)=a+w^{T}x\right\} }\sum_{i=1}^{n}R_{i}(f(X_{i})-Y_{i})^{2}.
\]

\end{enumerate}

\section{A family of simple AIPW estimators}

(Continuing the ``IPW estimator is not equivariant'' problem (1.3)
in Homework \#1.)

Suppose $\cd$ represents the dataset $(X_{1},R_{1},R_{1}Y_{1}),\ldots,(X_{n},R_{n},R_{n}Y_{n})$
from a MAR setting. For any $a\in\reals$, we'll write $\cd-a$ for
the dataset $(X_{1},R_{1},R_{1}\left(Y_{1}-a\right)),\ldots,(X_{n},R_{n},R_{n}\left(Y_{n}-a\right))$,
which is the same as $\cd$, but with each $Y$ value shifted by $a$.
Recall the following estimators:
\begin{eqnarray*}
\hat{\mu}_{\ipw}=\hat{\mu}_{\ipw}(\cd) & := & \frac{1}{n}\sum_{i=1}^{n}\frac{R_{i}Y_{i}}{\pi(X_{i})}\\
\hat{\mu}_{\ipw,a}=\hat{\mu}_{\ipw,a}(\cd) & := & \hat{\mu}_{\ipw}(\cd-a)+a,
\end{eqnarray*}
for any $a\in\reals$. In the last homework, we showed that 
\[
\hat{\mu}_{\ipw}(\cd-a)=\hat{\mu}_{\ipw}(\cd)-\frac{a}{n}\sum_{i=1}^{n}\frac{R_{i}}{\pi(X_{i})}
\]
 and that $\ex\hat{\mu}_{\ipw,a}(\cd)=\ex Y$. 
\begin{enumerate}
\item We can view $\hat{\mu}_{\ipw,a}$ as an augmented IPW (AIPW) estimator
-{}- that is, as a control-variate adjusted IPW estimator. With this
view, what is the control variate and what is its expectation?

 
\item Given what we learned about control variates, how would you choose
$a\in\reals$? (There are many reasonable answers to this question,
and I don't believe there is a single best answer without additional
assumptions. That said, the section on ``Optimal scaling to improve
variance'' in the control variates module may be a source of some
ideas.)

\end{enumerate}

\section{Election forecasting}

Suppose we want to forecast the outcome of an election with two candidates.
We have a budget to call $n$ people and ask who they'll vote for.
Each individual $i$ is described by the following random variables:
\begin{eqnarray*}
X_{i}\in\cx &  & \text{covariates describing individual \ensuremath{i}}\\
T_{i}\in\left\{ 0,1\right\}  &  & \text{indicator for whether \ensuremath{i} will vote in the election ("turnout indicator")}\\
R_{i}\in\left\{ 0,1\right\}  &  & \text{indicator for whether \ensuremath{i} will respond to a survey question if called}\\
Y_{i}\in\left\{ 0,1\right\}  &  & \text{indicator for which candidate an individual will vote for, if they vote}
\end{eqnarray*}
We'll assume the existence of an ``eligible voter generating distribution\footnote{In reality, there is a fixed set of potential voters. We're taking
the ``eligible voter generating distribution'' approach to align
more with the framework of our class. For large elections, the list
of all potential voters is so much larger than the size of the survey
sample that this is a very reasonable approximation.}'', and we'll refer to it as $P$. To carry out the survey, $n$
individuals are sampled from $P$. For individuals who respond (i.e.
for whom $R=1$), we will assume they reveal their true value of $Y$.
We'll write the full data corresponding to this scenario as
\[
\left(X,R,Y,T\right),\left(X_{1},R_{1},Y_{1},T_{1}\right),\ldots,\left(X_{n},R_{n},Y_{n},T_{n}\right),
\]
sampled i.i.d. from $P$. However, since we only observe $Y$ when
$R=1$, and we don't observe $T$ at all, we'll write the observed
data as 
\[
\left(X,R,RY\right),\left(X_{1},R_{1},R_{1}Y_{1}\right),\ldots,\left(X_{n},R_{n},R_{n}Y_{n}\right).
\]
\textbf{We'll make the following assumptions}:
\begin{enumerate}
\item $R$, $Y$, and $T$ are mutually independent given $X$. (In particular,
this implies $Y\indep R\mid X$ and $Y\indep T\mid X$.)
\item We have access to a function $\pi_{t}(x)=\pr\left(T=1\mid X=x\right)$
that gives the ``turnout probability'', i.e. the probability that
an individual will go vote, given their covariates\footnote{There are organizations and companies that produce this type of thing.
It's not a straightforward statistics or machine learning problem,
since it's not clear there are any high quality labels to fit a model
to. But we'll assume that somebody else has already solved this problem
for us.}.
\item We have access to a function $\pi_{r}(x)=\pr\left(R=1\mid X=x\right)$
that gives the ``response probability.'' This can function be estimated
using the observed data using, for example, logistic regression. But
we'll also assume that this part of the problem has already been solved
and we know $\pi_{r}(x)$. 
\item Every voter has at least some chance of responding to a survey. To
put this in mathematical terms: $\pi_{t}(x)>0\implies\pi_{r}(x)>0\quad\forall x\in\cx$. 
\end{enumerate}
To forecast the election, we want to estimate $\pr\left(Y=1\mid T=1\right)$,
i.e. the rate of voting for candidate $1$ among individuals who actually
go vote. In this problem, we'll use a variant of regression imputation
that accounts for the covariate shift between the survey respondent
distribution and the voter distribution.

\subsection{\label{subsec:Fitting-the-regression}Fitting the regression}

If we fit a model to the survey responses (i.e. the complete cases,
i.e. the $\left(X_{i},Y_{i}\right)$ pairs corresponding to $R_{i}=1$)
in the usual way (say empirical risk minimization over some space
of functions), we'll end up with a function $\hat{f}(x)$ that has
low risk with respect to the distribution $p(x,y\mid R=1)$. In other
words, $\hat{f}(x)$ will perform well for survey responders, but
what we really need is for $\hat{f}(x)$ to perform well for voters,
i.e. to have low risk w.r.t. the distribution $p(x,y\mid T=1)$. 
\begin{enumerate}
\item If we're fitting $\hat{f}(x)$ to data from $p(x,y\mid R=1)$ (without
importance weighting), then we expect $\hat{f}(x)\approx\ex\left[Y\mid X=x,R=1\right]$.
And if we could fit $\hat{f}(x)$ to data from $p(x,y\mid T=1)$ then
we would have $\hat{f}(x)\approx\ex\left[Y\mid X=x,T=1\right]$. Naturally,
you think that we'll want to try importance weighting to use data
from $p(x,y\mid R=1)$ to estimate $\ex\left[Y\mid X=x,T=1\right]$,
which is what we'd get with data from $p(x,y\mid T=1)$. But wait!
A colleague reminds you that we've assumed $Y\indep T\mid X$ and
$Y\indep R\mid X$, which implies $\ex\left[Y\mid X=x,R=1\right]=\ex\left[Y\mid X=x,T=1\right]=\ex\left[Y\mid X=x\right]$.
And so, your colleague claims that importance weighting doesn't make
a difference: we're estimating $\ex\left[Y\mid X=x\right]$ no matter
which data we're fitting on. Describe a circumstance when this claim
is reasonable and a circumstance when it is not reasonable\@. (Hint:
model misspecification)

\item Give an appropriate importance-weighted empirical risk estimate for
$f(x)$ in terms of a loss function $\ell\left(f(X),Y\right)$ and
the observed data described above. We'll only use it for learning
$\hat{f}$, so don't worry about scale factors.

\end{enumerate}

\subsection{Using our regression to forecast the election}

The goal of this section is come up with an estimator for $\pr\left(Y=1\mid T=1\right)$.
As noted in the introduction, this will be our forecast of the election
outcome.
\begin{enumerate}
\item Let $f(x)=\pr\left(Y=1\mid X=x,T=1\right)=\ex\left[Y\mid X=x,T=1\right]$.
Show that
\[
\pr\left(Y=1\mid T=1\right)=\frac{\ex\left[\pi_{t}(X)f(X)\right]}{\ex\left[\pi_{t}(X)\right]}.
\]
You can follow your own path, or use the steps in the subproblems
below.
\begin{enumerate}
\item Show that $\pr(Y=1\mid T=1)=\ex\left[f(X)\mid T=1\right]$. 

\item Show that $\ex\left[Tf(X)\right]=\pr\left(T=1\right)\ex\left[f(X)\mid T=1\right]$.
(Hint Remember that $T\in\left\{ 0,1\right\} $.)

\item Use the previous two results to show that $\pr(Y=1\mid T=1)=\ex\left[\pi_{t}(X)f(X)\right]/\pr\left(T=1\right)$.
(Hint: $\pi_{t}(X)=\ex\left[T\mid X\right]$.)

 
\item Conclude the proof of this section by showing that $\pr\left(T=1\right)=\ex\left[\pi_{t}(X)\right]$.

\end{enumerate}
\item Propose an estimator for $\pr\left(Y=1\mid T=1\right)$ that uses
an estimated regression function $\hat{f}(x)$ (such as the one developed
in Section \ref{subsec:Fitting-the-regression}) as a plug-in estimate
for $f(x)$, together with $\pi_{t}(x)$, $\pi_{r}(x)$, and a new
large sample\footnote{In the election context, getting samples of just covariates $X$ is
generally cheap compared to getting samples of $(X,Y)$ pairs.} $X_{1},\ldots,X_{N}$ of covariates from $P$. Your estimator should
converge to $\frac{\ex\left[\pi_{t}(X)\hat{f}(X)\right]}{\ex\left[\pi_{t}(X)\right]}$
as $N\to\infty$, though proving this is optional.

\end{enumerate}

\end{document}
