%% LyX 2.3.6 created this file.  For more info, see http://www.lyx.org/.
%% Do not edit unless you really know what you are doing.
\documentclass[ruled]{article}
\usepackage{courier}
\usepackage[T1]{fontenc}
\usepackage[latin9]{inputenc}
\usepackage[letterpaper]{geometry}
\geometry{verbose}
\setcounter{secnumdepth}{5}
\usepackage{color}
\usepackage{enumitem}
\usepackage{algorithm2e}
\usepackage{dsfont}
\usepackage{amsmath}
\usepackage{amssymb}
\usepackage[unicode=true,
 bookmarks=false,
 breaklinks=false,pdfborder={0 0 1},backref=section,colorlinks=true]
 {hyperref}

\makeatletter

%%%%%%%%%%%%%%%%%%%%%%%%%%%%%% LyX specific LaTeX commands.
\providecommand{\LyX}{\texorpdfstring%
  {L\kern-.1667em\lower.25em\hbox{Y}\kern-.125emX\@}
  {LyX}}

%%%%%%%%%%%%%%%%%%%%%%%%%%%%%% Textclass specific LaTeX commands.
\newlength{\lyxlabelwidth}      % auxiliary length 

\@ifundefined{date}{}{\date{}}
%%%%%%%%%%%%%%%%%%%%%%%%%%%%%% User specified LaTeX commands.
\definecolor{mygreen}{rgb}{0,0.6,0}
\definecolor{mygray}{rgb}{0.5,0.5,0.5}
\definecolor{mymauve}{rgb}{0.58,0,0.82}

\makeatother

\usepackage{listings}
\lstset{backgroundcolor={\color{white}},
basicstyle={\footnotesize\ttfamily},
breakatwhitespace=false,
breaklines=true,
captionpos=b,
commentstyle={\color{mygreen}},
deletekeywords={...},
escapeinside={\%*}{*)},
extendedchars=true,
frame=shadowbox,
keepspaces=true,
keywordstyle={\color{blue}},
language=Python,
morekeywords={*,...},
numbers=none,
numbersep=5pt,
numberstyle={\tiny\color{mygray}},
rulecolor={\color{black}},
showspaces=false,
showstringspaces=false,
showtabs=false,
stepnumber=1,
stringstyle={\color{mymauve}},
tabsize=2}
\begin{document}
\global\long\def\reals{\mathbf{\mathbb{R}}}%
\global\long\def\integers{\mathbf{Z}}%
\global\long\def\naturals{\mathbf{N}}%
\global\long\def\rationals{\mathbf{Q}}%
\global\long\def\ca{\mathcal{A}}%
\global\long\def\cb{\mathcal{B}}%
\global\long\def\cc{\mathcal{C}}%
\global\long\def\cd{\mathcal{D}}%
\global\long\def\ce{\mathcal{E}}%
\global\long\def\cf{\mathcal{F}}%
\global\long\def\cg{\mathcal{G}}%
\global\long\def\ch{\mathcal{H}}%
\global\long\def\ci{\mathcal{I}}%
\global\long\def\cj{\mathcal{J}}%
\global\long\def\ck{\mathcal{K}}%
\global\long\def\cl{\mathcal{L}}%
\global\long\def\cm{\mathcal{M}}%
\global\long\def\cn{\mathcal{N}}%
\global\long\def\co{\mathcal{O}}%
\global\long\def\cp{\mathcal{P}}%
\global\long\def\cq{\mathcal{Q}}%
\global\long\def\calr{\mathcal{R}}%
\global\long\def\cs{\mathcal{S}}%
\global\long\def\ct{\mathcal{T}}%
\global\long\def\cu{\mathcal{U}}%
\global\long\def\cv{\mathcal{V}}%
\global\long\def\cw{\mathcal{W}}%
\global\long\def\cx{\mathcal{X}}%
\global\long\def\cy{\mathcal{Y}}%
\global\long\def\cz{\mathcal{Z}}%
\global\long\def\pr{\mathbb{P}}%
\global\long\def\predsp{\cy}%
\global\long\def\outsp{\cy}%
\global\long\def\prxy{P_{\cx\times\cy}}%
\global\long\def\prx{P_{\cx}}%
\global\long\def\prygivenx{P_{\cy\mid\cx}}%
\global\long\def\ex{\mathbb{E}}%
\global\long\def\var{\textrm{Var}}%
\global\long\def\cov{\textrm{Cov}}%
\global\long\def\sgn{\textrm{sgn}}%
\global\long\def\sign{\textrm{sign}}%
\global\long\def\kl{\textrm{KL}}%
\global\long\def\law{\mathcal{L}}%
\global\long\def\eps{\varepsilon}%
\global\long\def\as{\textrm{ a.s.}}%
\global\long\def\io{\textrm{ i.o.}}%
\global\long\def\ev{\textrm{ ev.}}%
\global\long\def\convd{\stackrel{d}{\to}}%
\global\long\def\convp{\stackrel{P}{\to}}%
\global\long\def\eqd{\stackrel{d}{=}}%
\global\long\def\del{\nabla}%
\global\long\def\loss{V}%
\global\long\def\risk{R}%
\global\long\def\etal{\textrm{et. al.}}%
\global\long\def\tr{\operatorname{tr}}%
\global\long\def\trace{\operatorname{trace}}%
\global\long\def\diag{\text{diag}}%
\global\long\def\rank{\text{rank}}%
\global\long\def\linspan{\text{span}}%
\global\long\def\proj{\text{Proj}}%
\global\long\def\argmax{\operatornamewithlimits{arg\, max}}%
\global\long\def\argmin{\operatornamewithlimits{arg\, min}}%
\global\long\def\bfx{\mathbf{x}}%
\global\long\def\bfy{\mathbf{y}}%
\global\long\def\bfl{\mathbf{\lambda}}%
\global\long\def\bfm{\mathbf{\mu}}%
\global\long\def\calL{\mathcal{L}}%
\global\long\def\vw{\boldsymbol{w}}%
\global\long\def\vx{\boldsymbol{x}}%
\global\long\def\vxi{\boldsymbol{\xi}}%
\global\long\def\valpha{\boldsymbol{\alpha}}%
\global\long\def\vbeta{\boldsymbol{\beta}}%
\global\long\def\vsigma{\boldsymbol{\sigma}}%
\global\long\def\vmu{\boldsymbol{\mu}}%
\global\long\def\vtheta{\boldsymbol{\theta}}%
\global\long\def\vd{\boldsymbol{d}}%
\global\long\def\vs{\boldsymbol{s}}%
\global\long\def\vt{\boldsymbol{t}}%
\global\long\def\vh{\boldsymbol{h}}%
\global\long\def\ve{\boldsymbol{e}}%
\global\long\def\vf{\boldsymbol{f}}%
\global\long\def\vg{\boldsymbol{g}}%
\global\long\def\vr{\boldsymbol{r}}%
\global\long\def\vz{\boldsymbol{z}}%
\global\long\def\vk{\boldsymbol{k}}%
\global\long\def\va{\boldsymbol{a}}%
\global\long\def\vb{\boldsymbol{b}}%
\global\long\def\vv{\boldsymbol{v}}%
\global\long\def\vy{\boldsymbol{y}}%
\global\long\def\hil{\ch}%
\global\long\def\rkhs{\hil}%
\global\long\def\median{\text{Median}}%
\global\long\def\ipw{\text{ipw}}%
\global\long\def\ipwsn{\ensuremath{\text{ipwsn}}}%
\global\long\def\iw{\text{iw}}%
\global\long\def\myfoot#1{\let\thefootnote\relax\footnotetext{\tiny{#1}}}%
% Causal inference
\global\long\def\ate{\text{ATE}}%
\global\long\def\unif{\text{Unif}}%
\global\long\def\bern{\text{Bern}}%
\global\long\def\indep{\perp\!\!\! \perp}%
\global\long\def\ind#1{\mathds{1}\left[#1\right]}%
\global\long\def\se#1{\text{SE}\left(#1\right)}%
\global\long\def\corr{\text{Corr}}%
\global\long\def\expit{\text{expit}}%
\global\long\def\snipw{\ensuremath{\text{sn\_ipw}}}%
\global\long\def\bias{\ensuremath{\text{Bias}}}%
\global\long\def\sd{\textrm{SD}}%
\global\long\def\opt{\text{opt}}%
\global\long\def\dm{\ensuremath{\text{dm}}}%
\global\long\def\poem{\text{POEM}}%


\title{Tools and Techniques for Machine Learning\\
Homework 3}

\maketitle
\textbf{Instructions}: Your answers to the questions below, including
plots and mathematical work, should be submitted as a single PDF file.
It's preferred that you write your answers using software that typesets
mathematics (e.g. \LaTeX , \LyX , or Jupyter), though if you need
to you may scan handwritten work. For submission, you can also export
your Jupyter notebook and merge that PDF with your PDF for the written
solutions into one file. \textbf{Don't forget to complete the Jupyter
notebook as well, for the programming part of this assignment}. 

\section{Derviation of importance-weighted reward imputation}

Suppose we have a contextual bandit where context $X\in\cx$ has probability
density function $p(x)$ and reward vector $R\in\reals^{k}$ has conditional
distribution given by $P_{R\mid X}$. We want to use the direct method
to evaluate the performance of a static policy $\pi$. That is, we
want to use
\begin{eqnarray*}
\hat{V}_{\dm}(\pi) & = & \frac{1}{n}\sum_{i=1}^{n}\sum_{a=1}^{k}\hat{r}(X_{i},a)\pi(a\mid X_{i})\\
 & = & \frac{1}{n}\sum_{i=1}^{n}\ex_{A_{i}\sim\pi(\cdot\mid X_{i})}\left[\hat{r}(X_{i},A_{i})\right],
\end{eqnarray*}
where $\hat{r}(x,a)$ is some estimate for $\ex\left[R(A)\mid X=x,A=a\right]=\ex\left[R(a)\mid X=x\right]$
and 
\[
\left(X_{1},A_{1},R_{1}(A_{1})\right),\dots,\left(X_{n},A_{n},R_{n}(A_{n})\right)
\]
 is the logged bandit feedback from static policy $\pi_{0}$ on the
same contextual bandit distribution. The ``naive'' approach to fitting
$\hat{r}$ from some hypothesis space $\ch$ is least squares: 

\[
\hat{r}=\argmin_{r\in\ch}\frac{1}{n}\sum_{i=1}^{n}\left(r(X_{i},A_{i})-R_{i}(A_{i})\right)^{2}.
\]

\begin{enumerate}
\item With this approach, what is the covariate distribution in training?
Explain why we have a covariate shift between the train and target
distribution. 

\item Give an importance-weighted objective function $J(r)$ for finding
$\hat{r}$, and use the change of measure theorem to show that $\ex\left[J(r)\right]=\ex\left[r(X,A)-R(A)\right]^{2}$,
where $X\sim p(x)$, $R\mid X\sim P_{R\mid X}$ and $A\mid X\sim\pi(a\mid x)$.
In other words, the objective function is an unbiased estimate of
the expected square loss (i.e. the risk) of $r$ w.r.t. the target
distribution.

\end{enumerate}

\section{Optimizing 0/1 loss for binary classification}

\begin{enumerate}
\item Suppose we're trying to predict a binary event with outcome space
$\cy=\left\{ 0,1\right\} $. The action space is also $\ca=\left\{ 0,1\right\} $,
but we make randomized predictions with $\pr\left(A=1\right)=\pi$.
Suppose we know that $\pr\left(Y=1\right)=p$. We want to find $\pi\in\left[0,1\right]$
that minimizes our expected loss $\ex\ell\left(A,Y\right)$. What
is the optimal $\pi$ for the 0/1 loss: $\ell\left(a,y\right)=\ind{a\neq y}$?

\item Now consider the corresponding probabilistic prediction problem, where
the action space is $\ca=\left[0,1\right]$, which is supposed to
be a prediction for $\pr\left(Y=1\right)$. Give a loss function $\ell\left(\pi,y\right)$
for this action space, where $\pi\in\left[0,1\right]$, and outcome
space $\cy=\left\{ 0,1\right\} $ such that the expected loss is equivalent
to the expected loss of the previous problem. That is, find $\ell(\pi,y)$
such that $\ex_{Y\sim\text{Ber}(p)}\ell\left(\pi,Y\right)=\ex_{A\sim\text{Ber}(\pi),Y\sim\text{Ber}(p)}\ind{A\neq Y}$.
(Hint: Here and below, it may be helpful to note that if
\[
f(y)=\begin{cases}
a & \text{when }y=0\\
b & \text{when }y=1,
\end{cases}
\]
then $f(y)=a^{(1-y)}b^{y}$.)

\item Consider the conditional probability modeling setting with input space
$\cx=\reals^{d}$ and outcome space $\cy=\left\{ 0,1\right\} $. We
want to predict the probability of the outcome $1$ for any input
$x\in\cx$. The logistic regression model is $\pr\left(Y=1\mid X=x;w\right)=\phi(w^{T}x)$,
where $\phi(\eta)=1/\left(1+e^{-\eta}\right)$ (the standard logistic
function). We typically fit $w\in\reals^{d}$ by minimizing the negative
log-likelihood of $w$ for some data $\cd=\left((X_{i},Y_{i})\right)_{i=1}^{n}$
samplied i.i.d. from the data generating distribution $P$. The negative
log likelihood objective is 
\begin{eqnarray*}
J_{\text{nll}}(w) & = & -\left[\sum_{i=1}^{n}Y_{i}\log\phi(w^{T}X_{i})+\left(1-Y_{i}\right)\log\left(1-\phi(w^{T}X_{i})\right)\right].
\end{eqnarray*}
Now consider the setting of supervised learning with a $0/1$ loss
function: $\ell\left(a,y\right)=\ind{a\neq y}$. Give an expression
for the expected loss of a randomly sampled $(X,Y)\sim P$ when the
action $A\in\{0,1\}$ is drawn randomly from the logistic regression
model described above, namely $\pr\left(A=1\mid X=x;w\right)=\phi(w^{T}x)$

\item Give an objective function in terms of the data $\cd$ for finding
the $w$ that optimizes the expected loss you gave in the previous
problem\@. The objective function should be an unbiased estimate
for $\ex_{w}\ell(A,Y)$. Is this objective function equivalent to
the negative log-likelihood objective function described above? If
not, how might you expect the logistic regression models resulting
from the two different objective functions to compare? Consider the
case of large datasets and very expressive feature spaces. (Hint:
log loss is a proper scoring rule.)

\end{enumerate}

\end{document}
